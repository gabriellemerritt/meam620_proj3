\documentclass[english, twocolumn]{article}
%%%%%%%%%%%%%%%%%%%%%%%%%%%%%%%%%%%%%%%%%%%%%%%%%%%%%%%%%%%%%%%%%%%%%%%%%%%%%%%%%%%%%%%%%%%%%%%%%%%%%%%%%%%%%%%%%%%%%%%%%%%%
\usepackage{latexsym,amsmath,amssymb,amsfonts,fullpage,graphicx, float,adjustbox}
\usepackage{lastpage,listings}
\usepackage[hang, small,labelfont=bf,up,textfont=it,up]{caption} % Custom captions under/above floats in tables or figures
\usepackage{placeins}
\usepackage{adjustbox}
\usepackage[toc,page]{appendix}
\usepackage{tabulary}
\usepackage[margin=0.5in]{geometry}
\title{MEAM 620 Project 3} 
\author{Lou Lin, Mary Ibrahim, Gabrielle Merritt}
\begin{document}
\maketitle

\begin{abstract}
For our third project we created closed trajectories for a commercial drone ( Bebop ARDrone) using MATLAB to generate trajectories, and writing a C program to calculate and run our desired trajectories. 

\end{abstract}
\section*{Code Architecture }
For the code architecture, we used the Bebop's SDK basic flying sample as a base to send and recieve messgaes from the Bebop, and used it as a basis for designing our trajectory planning and control code. Figure~\ref{fig:code_draw} 
\section*{Methods}
\subsection*{Control System}
\paragraph{Estimating Flight States}
To have the Bebop fly any arbitrary trajectory in $R^3$, a position based controller is necessary. In order to implement a position based controller, we must be able to estimate the position and orientation of the quadrotor. Since the video transferred through wifi is not reliable, using epipolar geometry related technique on the computer's side is impossible. Luckily, the Bebop drone already have onboard attitude and velocity estimator using a combination of inertial sensors and optical flow. Assuming the sensor fusion onboard is reliable, it is reasonable for me to just pull the information from the drone registering my own callback functions in the API. \\
 
 \begin{adjustbox}{width=.45*\textwidth, height=4*\textheight,keepaspectratio}
\begin{lstlisting}
ARCOMMANDS_Decoder_SetCommonCommonStateBatteryStateChangedCallback
(batteryStateChangedCallback, deviceManager);

ARCOMMANDS_Decoder_SetARDrone3PilotingStateFlyingStateChangedCallback
(flyingStateChangedCallback, deviceManager);

ARCOMMANDS_Decoder_SetARDrone3PilotingStateAltitudeChangedCallback
(altitudeCallback, deviceManager);

ARCOMMANDS_Decoder_SetARDrone3PilotingStateAttitudeChangedCallback
(attitudeCallback, deviceManager);

ARCOMMANDS_Decoder_SetARDrone3PilotingStateSpeedChangedCallback
(velocityCallback, deviceManager);
\end{lstlisting}
\end{adjustbox}
The velocity measured is already set in the world frame. Since the Bebop can directly measure altitude, all it's left to do is to numerically integrate the velocity in the x and y direction to obtain my position. We used forward Euler approximation to calculate our position, and the result were reasonable. \\

\paragraph{Actuator Inputs}
We are able to send 4 different inputs to the Bebop drone. The
\begin{verbatim}
ARCOMMANDS_Generator_GenerateARDrone3PilotingPCMD()
\end{verbatim} function 
allows me to send desired roll and pitch angles, the yaw rate, and the velocity
in the world Z direction. The inner loop from propeller inputs to the desired
roll, pitch, yaw and velocity in Z is already close onboard, and we have no 
access to it unless we hack the firmware, so we will just assume that the inner
loop is stable and reacts fast to the reference signals. It is interesting that the
API has two different definitions of the body frame. The reference frame for
the actuator inputs has a rotation of $\pi$ about the X axis of the sensor
reference frame, therefore the controller needs to be modified accordingly.\\

\paragraph{Backstepping Control of position}
Leverging from project 1 phase 2, we can just take a controller in this form and expect the errors converge to zero exponentially.
$$a{i commanded} = a_{i desired} + K_{di}(V_{i desired} - V_i) + K_{p i}(r_{i desired} - r_i)$$

Then we can linearize about hover and use the commanded acceleration in the X and Y direction to find the desired roll and pitch angles. As long as the onboard attitude controller make these angles converge to the desired values much faster than my position controller, the quadrotor will perform reasonably.\\

However, the reference frames are defined differently. In the sensor frame, the Z axis points down, and the Z axis points up. In order to compensate for this inconsistency, I have to negate the pitch angle and the desired velocity in the Z direction.
$$
\phi_{commanded} = g(a_{x commanded}*\sin\psi - a_{y commanded})\cos\psi
$$
$$\theta_{commanded} = -g(a_{x commanded}*\cos\psi + a_{y commanded})\sin\psi$$


Note that I can only command desired Z velocity instead of a thrust. I must modify the control law for the commanded Z velocity (such is named gaz in the API).
\begin{equation}
gaz = V_{z desired} + K_{pz}(r_{z desired} - r_z)
\end{equation}

\subsection*{Trajectory Generation}
We used MATLAB to generate coefficients for quintic minimum jerk trajectories. For 
trajectories composed of successive straight line segments, each line was planned over  
seconds, with the quadrotor stopping at each waypoint.  We calculated coefficients  and  
in the x, y, and z directions based on the start and end positions of each line segment and 
start and end velocities and accelerations of zero. We then output the coefficients and 
length of time for each segment to a text file. After reading the text file into C, we 
calculated the time-parameterized position, velocity, and acceleration: 
These values were the desired positions, velocities, and accelerations that we fed into our 
controller. We used this method to generate a closed square trajectory. 

\subsubsection*{Curved trajectories}
For curved trajectories, we used parametric equations for x and y and had the parameter  
vary according to a quintic. We created a figure 8 trajectory in the x-y plane with the 
position equations: 
In MATLAB, we calculated coefficients  and  for  as it varies from zero to  and output 
these coefficients and the time to complete the entire trajectory to a text file. After 
reading these coefficients into C, we calculated  and and used these to calculate x and y 
positions, velocities, and accelerations, holding z constant.
\section*{Results} 

\section{\\Graphs} \label{App:AppendixA}


\end{document}
